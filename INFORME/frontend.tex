
\lstdefinelanguage{JavaScript}{
    keywords={break, case, catch, class, const, continue, debugger, default, delete, do, else, export, extends, finally, for, function, if, import, in, instanceof, let, new, return, super, switch, this, throw, try, typeof, var, void, while, with, yield, async, await},
    ndkeywords={true,false,null,undefined,NaN,Infinity},
    sensitive=true,
    comment=[l]{//},
    morecomment=[s]{/*}{*/},
    string=[b]",
    morestring=[b]',
    morestring=[b]`,
    showstringspaces=false,
    basicstyle=\ttfamily\small,
    keywordstyle=\color{blue}\bfseries,
    ndkeywordstyle=\color{orange}\bfseries,
    stringstyle=\color{teal},
    commentstyle=\color{gray}\itshape,
    numberstyle=\tiny\color{gray},
    numbers=left,
    stepnumber=1,
    numbersep=5pt,
    breaklines=true,
    breakatwhitespace=true,
    tabsize=2,
    frame=single,
    captionpos=b
}
\lstdefinelanguage{CSS}{
    keywords={color, background, margin, padding, font-size, font-family, display, position, top, left, right, bottom, border, width, height, text-align, justify-content, align-items},
    sensitive=true,
    comment=[s]{/*}{*/},
    morestring=[b]",
    morestring=[b]',
}


\section{FrontEnd}
El frontend de crumblr lo cree con ayuda de chatgpt en html y js con despliegue alpine en docker. Cabe destacar que el frontend es el mismo para la versión acoplada y desacoplada ya que me ahorraba tiempo y no podía crear más lambdas.

El código del frontend es el siguiente:
\subsection{Código}
\subsubsection{app.js}

Código JavaScript de la aplicación frontend que se comunica con la API de Crumblr para CRUD de crumbs.

\begin{lstlisting}[language=JavaScript, caption={Frontend JS para Crumbs}, label={lst:frontend_js}]
const API_URL = "https://v4vbkmeftj.execute-api.us-east-1.amazonaws.com/prod/crumbs";
const API_KEY = "x3SV9yBfwuWJ3M3NyT3E541ykzkEWdN9jrMj40ah";

async function fetchCrumbs() {
  const res = await fetch(API_URL, {
    headers: { "x-api-key": API_KEY }
  });

  if (!res.ok) {
    alert("Error al obtener los crumbs");
    return;
  }

  const data = await res.json();
  const list = document.getElementById("crumbs-list");
  list.innerHTML = "";

  data.forEach(crumb => {
    const li = document.createElement("li");
    li.innerHTML = `
      <div id="crumb-${crumb.crumb_id}">
        <p class="content">${crumb.content}</p>
        ${crumb.image_url ? `<img src="${crumb.image_url}" alt="Crumb image">` : ""}
        <small>${new Date(crumb.created_at).toLocaleString()}</small>
      </div>
      <button onclick="editCrumb('${crumb.crumb_id}', '${crumb.content.replace(/'/g, "\\'")}', '${crumb.image_url || ""}')"></button>
      <button onclick="deleteCrumb('${crumb.crumb_id}')"></button>
    `;
    list.appendChild(li);
  });
}

async function createCrumb(e) {
  e.preventDefault();
  const content = document.getElementById("content").value.trim();
  const image_url = document.getElementById("image_url").value.trim();

  if (!content) return alert("El contenido no puede estar vacío");

  const res = await fetch(API_URL, {
    method: "POST",
    headers: {
      "Content-Type": "application/json",
      "x-api-key": API_KEY
    },
    body: JSON.stringify({ content, image_url })
  });

  if (res.ok) {
    document.getElementById("create-form").reset();
    fetchCrumbs();
  } else {
    alert("Error al crear el crumb");
  }
}

async function deleteCrumb(id) {
  const res = await fetch(`${API_URL}/${id}`, {
    method: "DELETE",
    headers: { "x-api-key": API_KEY }
  });

  if (res.ok) fetchCrumbs();
  else alert("Error al eliminar el crumb");
}

function editCrumb(id, currentContent, currentImage) {
  const newContent = prompt("Editá el contenido del crumb:", currentContent);
  if (newContent === null) return;

  const newImage = prompt("Editá la URL de imagen (o dejá vacío):", currentImage);
  updateCrumb(id, newContent, newImage);
}

async function updateCrumb(id, content, image_url) {
  const res = await fetch(`${API_URL}/${id}`, {
    method: "PUT",
    headers: {
      "Content-Type": "application/json",
      "x-api-key": API_KEY
    },
    body: JSON.stringify({ content, image_url })
  });

  if (res.ok) {
    fetchCrumbs();
  } else {
    alert("Error al actualizar el crumb");
  }
}

document.getElementById("create-form").addEventListener("submit", createCrumb);
fetchCrumbs();
\end{lstlisting}

Un detalle importante es que a la hora de lanzar este código es necesario cambiar la URL de la API y la API KEY por las correspondientes a la implementación del backend.
A pesar de haber implementado apiurl y apikey en el yaml del frontend, no conseguí cargarlos correctamente y decidí dejarlos hardcodeados para llegar a la entrega.

\subsubsection{index.html}

\begin{lstlisting}[language=HTML, caption={Interfaz web de Crumblr}, label={lst:index_html}]
<!DOCTYPE html>
<html lang="es">
<head>
  <meta charset="UTF-8">
  <title>Crumblr</title>
  <link rel="stylesheet" href="style.css">
</head>
<body>
  <h1>Crumblr</h1>

  <section id="create">
    <h2>Nuevo Crumb</h2>
    <form id="create-form">
      <textarea id="content" placeholder="Que estas pensando?" required></textarea>
      <input type="text" id="image_url" placeholder="URL de imagen (opcional)">
      <button type="submit">Publicar</button>
    </form>
  </section>

  <section id="list">
    <h2>Crumbs recientes</h2>
    <ul id="crumbs-list"></ul>
  </section>

  <script src="app.js"></script>
</body>
</html>
\end{lstlisting}

\subsubsection{style.css}
\begin{lstlisting}[language=CSS, caption={Estilos CSS para Crumblr}, label={lst:style_css}]
    body {
  font-family: system-ui, sans-serif;
  background: #fafafa;
  color: #333;
  padding: 2rem;
  max-width: 600px;
  margin: auto;
}

h1 {
  text-align: center;
  color: #222;
}

form {
  display: flex;
  flex-direction: column;
  gap: 0.5rem;
  margin-bottom: 2rem;
}

textarea, input, button {
  padding: 0.75rem;
  border-radius: 10px;
  border: 1px solid #ccc;
  font-size: 1rem;
}

button {
  background: #ffb703;
  color: #222;
  font-weight: bold;
  border: none;
  cursor: pointer;
}

button:hover {
  background: #fb8500;
}

ul {
  list-style: none;
  padding: 0;
}

li {
  background: white;
  border-radius: 10px;
  margin-bottom: 1rem;
  padding: 1rem;
  box-shadow: 0 2px 5px rgba(0,0,0,0.1);
  display: flex;
  justify-content: space-between;
  align-items: center;
}

li img {
  max-width: 100%;
  border-radius: 10px;
  margin-top: 0.5rem;
}

small {
  color: #888;
  display: block;
  margin-top: 0.25rem;
  font-size: 0.8rem;
}
\end{lstlisting}

\subsection{dockerfile}
De nuevo, intento mantenerlo simple con alpine.

\begin{lstlisting}[language=Dockerfile, caption={Dockerfile para Frontend}]
FROM nginx:stable-alpine

RUN rm -rf /usr/share/nginx/html/*

COPY . /usr/share/nginx/html

COPY nginx.conf /etc/nginx/conf.d/default.conf

EXPOSE 80

CMD ["nginx", "-g", "daemon off;"]
\end{lstlisting}