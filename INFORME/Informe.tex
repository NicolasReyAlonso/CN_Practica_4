\documentclass[12pt,a4paper]{article}
\usepackage[utf8]{inputenc}
\usepackage[spanish]{babel}
\usepackage{geometry}
\usepackage{hyperref}
\usepackage{listings}
\usepackage{xcolor}

\geometry{margin=2.5cm}

\title{Ejercicio entregable\\Computación en la Nube}
\author{Nicolás Rey Alonso}
\date{\today}

\begin{document}

\maketitle

\include{index}
\include{acoplado}

\section{Introducción}
En este informe se detallan las actividades realizadas, centradas en el despliegue de infraestructura y servicios en AWS mediante 
\textbf{CloudFormation}. Se describen los trabajos tanto de manera desacoplada como el trabajo en curso de integración/acoplamiento.

\section{Trabajo realizado: Desacoplado}

Se han creado varias plantillas YAML para desplegar de manera independiente los componentes del proyecto \textit{Crumblr}, separando frontend y backend:

\begin{itemize}
    \item \textbf{Repositorios de imágenes:} Dos plantillas YAML para la creación de los repositorios en Amazon ECR, uno para el frontend y otro para el backend.
    \item \textbf{Backend:} Una plantilla YAML que despliega un \textbf{ECS cluster} con la \textbf{task definition} correspondiente al backend, incluyendo la configuración de API Gateway para exponer los endpoints de la aplicación.
    \item \textbf{Frontend:} Una plantilla YAML que despliega un \textbf{ECS cluster} para el frontend, configurando el \textbf{Application Load Balancer} y el security group necesario.
\end{itemize}

Estas configuraciones permiten que cada componente funcione de manera autónoma y puedan probarse independientemente.

\section{Trabajo en progreso: Acoplado}

Actualmente se está trabajando en la integración de los servicios frontend y backend, buscando que ambos puedan comunicarse correctamente mediante la API creada. Este paso incluye:

\begin{itemize}
    \item Configuración de las variables de entorno en las task definitions para que el frontend pueda acceder a la API del backend.
    \item Ajustes de CORS en API Gateway para permitir peticiones desde el dominio del frontend.
    \item Pruebas de comunicación y validación del flujo completo de creación, lectura, actualización y eliminación de ``crumbs`` desde la interfaz del frontend.
\end{itemize}

\section{Conclusiones}

Hasta ahora se ha logrado:
\begin{itemize}
    \item Desplegar de manera independiente los componentes del proyecto en AWS utilizando CloudFormation.
    \item Configurar backend y frontend en ECS Fargate con sus respectivos load balancers.
    \item Probar la comunicación básica entre frontend y backend mediante API Gateway.
\end{itemize}

El siguiente paso será completar la integración completa (acoplamiento) y asegurar que la aplicación funcione de manera coherente como un sistema completo.

\end{document}
